\documentclass{article}
\usepackage{graphicx} % Required for inserting images

\usepackage[utf8]{inputenc}
\usepackage[T5]{fontenc}
\usepackage[vietnamese]{babel}       % Required for Vietnamese

\title{DSA}
\author{Tôn Huỳnh Chí}
\date{February 2025}

\begin{document}
\maketitle

\section {Hashing}

\textbf{Definition:} \textit{Hashing is the process of mapping a key to a address for storing and retrieving.}\\
\textbf{Advantages:} \textit{Easy to implement.}\\
\textbf{Disadvantages:} \textit{Collisions.}\\

\textbf{Collision Handling:}

\begin{itemize}
    \item \textbf{Linear probing} \textit{Find the next empty slot.}
        \begin{itemize}
            \item \textbf{Advantages:} \textit{Easy to implement.}
            \item \textbf{Disadvantages:} \textit{Clustering, approximate to O(N) in the worst case.}
        \end{itemize}
    \item \textbf{Chaining:} \textit{Insert at the end of the linked list.}
        \begin{itemize}
            \item \textbf{Advantages:} \textit{No clustering.}
            \item \textbf{Disadvantages:} \textit{Extra space, hard to balance the size of linked-lists in hash table.}
        \end{itemize}
    \item \textbf{Quadratic probing:} \textit{Find the next empty slot by quadratic function.}
        \begin{itemize}
            \item \textbf{Advantages:} \textit{No clustering.}
            \item \textbf{Disadvantages:} \textit{May not find empty slot, lead to infinite loop.}
        \end{itemize}
    \item \textbf{Double hashing:} \textit{Find the next empty slot by another hash function.}
        \begin{itemize}
            \item \textbf{Advantages:} \textit{No clustering.}
            \item \textbf{Disadvantages:} \textit{Complex, need a large size hash table to work well.}
        \end{itemize}
\end{itemize}

\section{Recursion}

\textbf{Definition:} \textit{Recursion is repetition, a funtion will invokes itself.}\\
\textbf{Advantages:} \textit{Easy to implement.}\\
\textbf{Disadvantages:} \textit{When a function is called, a stack frame is push onto the stack, if recursion many times, lead to stack-over-flow.}\\


\end{document}